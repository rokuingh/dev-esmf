%===============================================================================
% CVS $Id: append_testing.tex,v 1.3 2006/12/13 20:19:07 cdeluca Exp $
% CVS $Source: /cvsroot/esmf/esmf/src/doc/dev_guide/append_testing.tex,v $
% CVS $Name: MESH_BRANCH $
%===============================================================================

\section*{Appendix A:  Testing Terminology}

Industry-accepted definitions exist for software errors and defects (faults), found in the ANSI/IEEE, 
Glossary of Software Engineering Terminology, are
listed in Table A1. \\

Table A1 -- IEEE Software Engineering Terminology \cite{ieee} \\

\begin{tabular}{|p{1.3in}|p{4.7in}|} \hline

{\bf Category}  &    {\bf Definition} \\ \hline \hline
Error    	&    The difference between a computed, observed, or measured value or
                     condition and the true, specified, or theoretically correct value 
                     or condition. \\ \hline
Fault    	&    An incorrect step, process, or data definition in a computer program. \\ \hline
Debug   	&    To detect, locate, and correct faults in a computer program. \\ \hline 
Failure  	&    The inability of a system or component to perform its required functions 
              	     within specified performance requirements.  It is manifested as a fault.  \\ \hline
Testing  	&    The process of analyzing a software item to detect the differences between 
              	     existing and required conditions (that is, bugs) and to evaluate the features 
                     of the software items. \\ \hline
Static analysis &    The process of evaluating a system or component based on its form,
                     structure, content, or documentation. \\ \hline
Dynamic analysis &   The process of evaluating a system or component based on its behavior
                     during execution. \\ \hline
Correctness 	&    $\bullet$ The degree to which a system or component is free from faults in its
                     specification, design, and implementation.  \newline
                     $\bullet$ The degree to which software, documentation, or other items meet
                     specified requirements.  \newline
                     $\bullet$ The degree to which software, documentation, or other items meet user
                     needs and expectations, whether specified or not. \\ \hline
Verification    &    $\bullet$ The process of evaluating a system or component to determine whether the
                     products of a given development phase satisfy the conditions imposed at
                     the start of that phase. \newline 
                     $\bullet$ Formal proof of program correctness. \\ \hline
Validation      &    The process of evaluating a system or component during or at the end of
                     the development process to determine whether it satisfies specified
                     requirements. \\ \hline
\end{tabular}










