% $Id$

%\subsection{Design and Implementation Notes}

\begin{enumerate}

\item{\bf Class and directory hierarchy.}
The LocalArray class is an internal class which is not 
visible outside the framework.
As described below it contains all the information needed to
map memory into a multidimensional array.  It is used internally
by other parts of the framework to provide a uniform interface to 
internal data.  The public Array class
adds the domain information needed to support simple communication of
subdomains.  The communication routine source for operations such
as halo and regrid are separated out into a separate directory 
to allow Arrays to be used internally before all the 
communication code has been compiled.

\item{\bf LocalArray design.}
The purpose of the LocalArray class is to fully describe 
a homogeneous
multidimensional array, possibly strided, so that it can be understood
and manipulated by multiple languages.   It describes the
relationship between array indices and the linear form of the array
in physical memory.  It describes all dimensions which are present;
there are no hidden or implied dimensions.  The first dimension specified
is always the one which varies fastest in linearized memory regardless of
interface language used to create or access the array.

The LocalArray type is defined separately because it is used by the
Fortran code in Fields, IGrids, Route, and Regrid to refer to data 
independent of Type/Kind/Rank differences.  This abstraction removes
the need for these other objects to provide
heavily-overloaded interface blocks to hide the number of
different data combinations supported by these routines.

The metadata in this class would be unnecessary for a straight
Fortran implementation since the language provides methods for querying
arrays for this information.  But for interoperability between different
versions of Fortran, different hardware architectures, 
and the C++ interfaces
it is necessary to keep the information in a format which can be
easily managed by the ESMF and not buried in the language layer.


\item{\bf Array design.}
The purpose of the Array class is to support all the functions of
the LocalArray class plus domain information to support halo,
regrid, and data redistribution operations.

The create routine in C++ requires the user to supply all values for
rank, shape, strides, etc because there are no language constructs which
allow a pointer to be queried for this information.

There are two types of create routines in Fortran.  One mimics the C++
interface and requires the user to supply all information.  
The second, which is expected to be more useful and natural, 
is simply passed an existing Fortran array pointer.  Most of the
array attributes can be queried using language-defined functions which
should be portable to any Fortran compiler.   If other attributes are needed
which require compiler-dependent code, the implementation approach
will be to write specific platform-dependent code for the most common
compilers and platforms, and then use less efficient or more indirect 
methods for obtaining this information which will be the default if 
no compiler-specific method has been written.

The major challenge for the Array class implementation is that it 
contains user data which can be of many different types, kinds,
and ranks, each of which is a different type in Fortran 90 and
the language is strictly typechecked.

For the C++ interface polymorphism and templates can ease the burden of 
maintaining the interface; in Fortran the interface to the user
is simplified by using interface blocks but the number of internal
routines will be quite large.  Judicious use of the macro preprocessor
allows generic routines to be expanded on a per-datatype basis.

Another way the data interfaces will be kept down to a manageable
size is to explicitly limit the number of supported user datatypes to:
\begin{description}
\item[integer*1/byte]
\item[integer*2/short]
\item[integer*4/int]
\item[integer*8/long]
\item[real*4/float]
\item[real*8/double]
\end{description}


\item{\bf ArrayComm design.}
The source of the ESMF\_Array communication routines are separated out into 
an ESMF\_ArrayComm directory.  This allows basic Array functions to be
compiled and used by IGrid, Field, and FieldBundle code before the communication
code has been compiled.  

The Array communication routines require IGrid and DataMap information in 
addition to the Array itself.   The ArrayDataMap is needed to identify which
axes in the Array correspond to IGrid axes.  The IGrid is needed to compute
where in the overall object this Array is located and which pieces of
the overall object are located on which DE.


\end{enumerate}





