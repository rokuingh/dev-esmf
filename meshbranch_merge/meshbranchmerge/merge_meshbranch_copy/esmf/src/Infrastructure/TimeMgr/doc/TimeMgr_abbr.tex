\newpage
\begin{center}
\begin{table}
\caption{\label{time-abbrevs}Specifiers for Times and TimeIntervals.}
\begin{tabular}{|p{1in}|p{3.5in}|}
\hline
Name & Meaning \\
\hline\hline
{\bf YR} & Integer year. \\
\hline
{\bf MM} & Integer month of year. \\
\hline
{\bf D} & Integer number of days. \\
\hline
{\bf d} & Floating point number of days. \\
\hline
{\bf H} & Integer number of hours. \\
\hline
{\bf h} & Floating point number of hours. \\
\hline
{\bf M} & Integer number of minutes. \\
\hline
{\bf m} & Floating point number of minutes. \\
\hline
{\bf S} & Integer number of seconds. May need $_{nd}$ form.\\
\hline
{\bf s} & Floating point number of seconds. \\
\hline
{\bf MS} & Integer number of milliseconds. \\
\hline
{\bf ms} & Floating point number of milliseconds. \\
\hline
{\bf TS} & Integer number of 1/10,000 seconds. \\
\hline
{\bf US} & Integer number of microseconds. \\
\hline
{\bf NS} & Integer number of nanoseconds. \\
\hline
{\bf DD} & Day of month. \\
\hline
{\bf O} & Time zone offset in integer number of hours and minutes. \\
\hline
{\bf \_nd} & Suffix to indicate {\bf $ {\rm integer} + \frac{n}{d}$} form,
where $n$ and $d$ are integers. For example, {\bf S\_nd} has an integer
second component and a fractional second component. {\bf \_nd} provides 
a mechanism for supporting exact behavior.
\\
\hline
\end{tabular}
\end{table}
\end{center}






