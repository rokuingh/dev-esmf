% $Id$

Many of the components that will run and interact within the ESMF
are prognostic simulation and data assimilation codes employing time-stepping 
approaches to solve a numerical implementation of a set of mathematical 
equations. Coordinating component to component interactions and coordinating 
interactions between components and external systems, from which data is 
ingested or to which data is exported, requires precise notions of time. The
role of the ESMF Time Manager (\funcname) is to provide a standard set 
of \char`\"{}time services\char`\"{}
to components running under ESMF. The ESMF time management concepts
have many parallels with real alarm clocks and with modern electronic
scheduling systems. However, ESMF applications have
a number of special requirements that go beyond the facilities that
have become standard in mainstream software. The \funcname is
designed to meet both the conventional needs and specialized demands
of Earth system applications.

\subsection{Location}

The time management set of \char`\"{}time services\char`\"{}
will be part of the ESMF Infrastructure layer. Including these services
within the ESMF Infrastructure layer permits

\begin{itemize}
\item development of multiple components with compatible notions of time;
\item development of a robust, core library of common time functions;
\item subsequent community development of common higher-level time operations.
\end{itemize}

\subsection{Scope}

The time manager is not intended to be a comprehensive set of services for
all time related operations. For example, it will not include a complete
time-zone capability that supports translation between all of Earth's 
more than 300 timezones. Nor will it include
algorithms that calculate detailed orbital quantities such as perihelion, 
obliquity and precession.
However, the time manager will provide a generic foundation for developing 
libraries that do provide such custom, specialized time services. It 
will support this in a way that permits easy interoperability and 
code sharing. As such, the time manager is anticipated to be a library of 
software that is targeted at both ESMF component developers and at specialized 
library developers.





