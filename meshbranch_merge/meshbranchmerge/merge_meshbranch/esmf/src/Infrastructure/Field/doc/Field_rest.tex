% $Id$

%\subsubsection{Restrictions and Future Work}

\begin{enumerate}
\label{Field:rest}

\item {\bf CAUTION:} It depends on the specific entry point of {\tt ESMF\_FieldCreate()} used during Field creation, which Fortran operations are supported on the Fortran array pointer {\tt farrayPtr}, returned by {\tt ESMF\_FieldGet()}. Only if the {\tt ESMF\_FieldCreate()} {\em from pointer} variant was used, will the returned {\tt farrayPtr} variable contain the original bounds information, and be suitable for the Fortran {\tt deallocate()} call. This limitation is a direct consequence of the Fortran 95 standard relating to the passing of array arguments.

\item {\bf No mathematical operators.}  The Fields class does not 
currently support advanced
operations on fields, such as differential or other
mathematical operators.

\item {\bf No vector Fields.}  ESMF does not currently  support storage of 
multiple vector Field components in the same Field component, although
that support is planned.  At this time users need to create a 
separate Field object to represent each vector component.

\item {\bf Conservative Regridding} 
The conservative regridding is not designed to 
prevent diffusion of mass.  The L2 projection method is {\it globally} constrained
for conservation of mass.  This means that the mass will be conserved to high precision
over the entire regridding domain, but this conservation will not be restricted
in a local sense, and mass may be diffused across a large stencil.  

The conservative regridding can also have problems 
with high interpolation error in regions where the destination grid is of higher 
resolution than the source grid.  It is expected that this effect is seen in select
sub domains of the regridding where the
grid resolution difference is not constant throughout the entire domain.  In domains
where the destination grid resolution is higher than the source grid resolution some
points can be left out of the mapping or not mapped with the expected source point 
contributions.  This is because the destination points depend on the source points 
which lie in the {\it neighboring} destination cells.  Therefore, if there are 
several destination cells contained within a single source cell, cases will arise
where the destination points are incorrectly mapped, if they are mapped at all.

\end{enumerate}
