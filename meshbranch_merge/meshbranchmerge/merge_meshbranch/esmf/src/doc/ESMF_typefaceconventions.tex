
The following conventions for fonts and capitalization are used
in this and other ESMF documents. \newline

\begin{tabular}{lll}
{\bf Style} & {\bf Meaning} & {\bf Example} \\ \hline
{\it italics}  & documents & {\it ESMF Reference Manual}\\
{\tt courier}  & code fragments & {\tt ESMF\_TRUE}\\
{\tt courier()}  & ESMF method name & {\tt ESMF\_FieldGet()} \\
{\bf boldface} & first definitions & An {\bf address space} is ...\\
{\bf boldface} & web links and tabs & {\bf Developers} tab on the website \\
{Capitals}     & ESMF class name & DataMap \\
\end{tabular} 
 
ESMF class names frequently coincide with words commonly
used within the Earth system domain (field, grid, component, array, 
etc.)  The convention we adopt in this manual is that if a word is 
used in the context of an ESMF class name it is capitalized, and 
if the word is used in a more general context it remains in lower 
case.  We would write, for example, that an ESMF Field class 
represents a physical field.  

Diagrams are drawn using the Unified Modeling Language (UML).  UML 
is a visual tool that can illustrate the structure of 
classes, define relationships between classes, and describe sequences
of actions.  A reader interested in more detail can refer to a 
text such as {\it The Unified Modeling Language Reference Manual.}
 \cite{uml}

