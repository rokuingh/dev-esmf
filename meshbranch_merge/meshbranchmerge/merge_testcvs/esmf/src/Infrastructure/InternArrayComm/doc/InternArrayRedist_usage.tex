Redist is designed to be called with Field or FieldBundle arguments in order to
utilize information embedded in these objects.  For example, Redist requires
knowledge of the distribution contained in the underlying IGrid and of the
relative location (staggering) of Fields on the IGrid.  In addition, Redist uses
any mask information that may be associated with a Field.  However, ESMF also
provides an Array interface for users who have gathered all necessary
information.

In general, Redist interfaces are relatively simple and require little
information directly from users.  The only option currently available to users
via Redist interfaces sets communication strategy through an optional argument,
{\tt routeOptions}, and is not normally specified.  For more information on
this option, please see Section~\ref{opt:routeopt}.

Like other high-level ESMF communication methods, Redist has separate
functions for RedistStore, Redist, and RedistRelease.  The Store functions
initialize and precompute the communication patterns required for performing
the data redistribution, returning an object called an {\tt ESMF\_RouteHandle}.
This object is reusable by other co-located ESMF data objects.
The Redist functions use the communication patterns contained in the RouteHandle
object to perform the actual redistribution of the ESMF data objects.  The
Release function deletes the RouteHandle object and frees all memory associated
with a Redist.  ESMF assumes users will call each of these functions
sequentially, but the Redist also has Field and FieldBundle interfaces that allow for
a single call to the Redist function instead, without requiring Store and
Release calls.  In this case, the Redist interfaces have been overloaded and
for this version a parent VM must be included in the calling argument list.
Also, no RouteHandle is returned to the user for possible reuse.

Please see the examples below for representative FieldRedist usage.



