
\label{sec:glos}

This glossary defines terms used in Earth system modeling to describe 
parallel computer architectures, grids and grid decompositions, and 
numerical and computational methods.

\begin{description}

\label{glos:360DayCal}
\item[360-day calendar] A calendar in which 
  every one of twelve months has thirty days.  See also \htmlref{Calendar}
  {glos:Calendar}, \htmlref{no-leap calendar}{glos:NoLeap}.

\label{glos:Accumulator}
\item[Accumulator] A facility for collecting 
  and averaging data values.  Generally accumulators are associated with 
  temporal averaging, although they might be associated with 
  other weighted averaging operations.  ESMF does not yet have accumulators.

\label{glos:API}
\item[Application Programming Interface (API)] API refers to the 
  set of routines and types in a software package that are
  available to its users.  It doesn't include private or internal
  routines or types.
  
\label{glos:Alarm}
\item[Alarm] Like a real alarm clock, the ESMF Alarm class notifies
  the user of an event that occurs at a particular 
  time (or set of times).  In order to determine whether it
  is "ringing", an ESMF Alarm is ``read'' by an explicit
  application action.  An Alarm is associated
  with a particular \htmlref{Clock}{glos:Clock}.

\label{glos:Application}
\item[Application] A coherent computational 
  entity run as a single executable or set of communicating executables.  
  It typically consists of a set of interacting components.  
  See also \htmlref{component}{glos:Component}.

\label{glos:Array}
\item[Array] An ESMF class that represents a multi-dimensional
  data array.  Unlike a native Fortran or C++ array, an ESMF Array
  can store information about halo points.  See also \htmlref{halo}{glos:Halo}.

\label{glos:BackGrid}
\item[Background grid] 
  A background grid associates each point in an observational data stream 
  (Location Stream) with a location on a grid. A single grid cell may contain 
  zero or more Location Stream points. See also \htmlref{Location Stream}{glos:LocStream}, \htmlref{cell}{glos:Cell}. 

\label{glos:BUFR}
\item[BUFR]
  Binary Universal Form of Representation.  This is a World Meteorological Organization
data format.  See \htmladdnormallink{BUFR links}{http://dss.ucar.edu/docs/formats/bufr/}.

\label{glos:FieldBundle}
\item[FieldBundle] The ESMF FieldBundle class represents a set of fields that 
  are associated with the same physical grid and are distributed in 
  the same fashion across the same physical axes.  Fields within a
  FieldBundle may be staggered differently and may have different (non-distributed)
  dimensions.  See also \htmlref{Field}{glos:Field}, \htmlref{Packed FieldBundle}{glos:PackedFieldBundle},
  \htmlref{Loose FieldBundle}{glos:LooseFieldBundle}. 

\label{glos:Calendar}
\item[Calendar] The Calendar is an ESMF class that 
  stores a representation of a particular calendar type, such as Gregorian.
  See also specific calendar types such as \htmlref{360-day}
  {glos:360DayCal} and \htmlref{no-leap}{glos:NoLeap}.

\label{glos:Cell}
\item[Cell] A physical location that is specified by both 
  its extent (vertices) and nominal central location, and is associated with 
  a single integer index value or a set of integer index values ( e.g.
  (i) for 1-d, (i,j) for 2-d, (i,j,k) for 3d ). 
  See also \htmlref{index}{glos:Index}.

\label{glos:CF}
\item[CF Conventions]
Climate and Forecast Conventions.  These are emerging conventions for expressing 
Earth science metadata.  See the \htmladdnormallink{CF home page}{http://www.cgd.ucar.edu/cms/eaton/cf-metadata/}.

\label{glos:CRB}
\item[Change Review Board (CRB)] The Change Review Board is the ESMF management
body that sets project schedules and priorities.  Its Terms of Reference
are in the \htmladdnormallink{ESMF Project Plan}{http://www.earthsystemmodeling.org/management/}.

\label{glos:Clock}
\item[Clock] Clock is an ESMF class that tracks the passage of time and 
  reports the current time instant.  An ESMF Clock 
  is stepped forward in increments of a time step, and can be associated
  with one or more Alarms.
  See also \htmlref{Time}{glos:TimeInstant}, \htmlref{Time Interval}{glos:TimeInterval}, \htmlref{Alarm}{glos:Alarm}.

\label{glos:Component}
\item[Component] The ESMF Component class represents large-scale
  computational entities associated with a particular physical
  process or computational function, such as a land model.  
  Currently ESMF supports \htmlref{Gridded Component}{glos:GridComp}
  and \htmlref{Coupler Component}{glos:Coupler} classes.
  Components may be \htmlref{generic}{glos:GenericComp} or 
  \htmlref{user-supplied}{glos:UserComp}.  

\label{glos:CompDomain}
\item[Computational domain] For a given DE, the data points that 
  have unique global indices and are updated by the DE.  
  See also \htmlref{exclusive domain}{glos:ExcDomain}, \htmlref{total domain}
  {glos:TotDomain}, \htmlref{halo}{glos:Halo}.

\label{glos:CompResource}
\item[Computational resource] Something that appears as a
  physical or virtual computer resource. Example of computational resources
  are a CPU, a network connection, a communication API, a protocol, a 
  particular network fabric or a piece of computer memory. 

\label{glos:ConcurrentExecution}
\item[Concurrent execution] 
  Concurrent execution of model components occurs when two or more components,
  whether in the same or different executables, run simultaneously.
  See also \htmlref{Sequential execution}
  {glos:SequentialExecution}.

\label{glos:Congruent}
\item[Congruent] 
  If all Fields in a FieldBundle contain the same data type, rank, shape, and 
  relative locations, the FieldBundle is said to be congruent. 

\label{glos:Coupler}
\item[Coupler Component]
  An ESMF Component that includes all data and actions needed to enable 
  communication between two or more Gridded Components. 
  See also \htmlref{component}{glos:Component}, \htmlref{Gridded Component}
  {glos:GridComp}. 

\label{glos:CurvilinearGrid}
\item[Curvilinear grid]
  A curvilinear grid is a logically rectangular grid in which 
  coordinates in physical space must be specified by giving 
  the explicit coordinates for each point.  Curvilinear
  grids are often uniform or rectilinear grids that have been 
  warped, for example in order to place a pole over land points
  so it does not affect the computations performed on an
  ocean model grid.  See also 
  \htmlref{logically rectangular grid}{glos:LogRectGrid},
  \htmlref{Uniform grid}{glos:UniformGrid}, \htmlref{Rectilinear grid}
  {glos:RectilinearGrid}.  
  
\label{glos:DataDep}
\item[Data dependency] The property of a computational
  operator that defines the data indices required to perform
  the computation at a point.  

\label{glos:DataParallel}
\item[Data parallel] The quality of an application that allows roughly 
  the same calculation to be performed by all processors at the same 
  time on the same data set, which is partitioned among multiple memory 
  locations.  Single components that do not contain nested components
  are often data parallel.  See also 
  \htmlref{task parallel}{glos:TaskParallel}, \htmlref{SPMD}{glos:SPMD}, 
  \htmlref{MPMD}{glos:MPMD}.   

\label{glos:DataTranspose}
\item[Data transpose] Rearrangement of data arrays that share the 
  same \htmlref{global domain}{glos:GlobDomain}.  

\label{glos:DayOfYear}
\item[Day of year] The day number in the calendar year. 
  January 1 is day 1 of the year. Day of year expressed in a floating point 
  format is used to express the day number plus the time of day. 
  For example, assuming a Gregorian calendar:

\begin{tabular}{ll}
  {\bf date}              & {\bf day of year} \\
  \hline 
  10 January 2000, 6Z     & 10.25 \\
  31 December 2000, 18Z   & 366.75 
\end{tabular}

\label{glos:DE}
\item[DE] 
  Short for \htmlref{Decomposition Element}{glos:Decomp_Element}.

\label{glos:DELayout}
\item[DELayout] DELayout is the ESMF class that
  defines the topology of a set of DEs and specifies 
  how the DEs are assigned to PETs in an ESMF  
  \htmlref{Virtual Machine}{glos:VM}. 

\label{glos:Decomp_Element}
\item[Decomposition Element (DE)]
  A DE is the smallest unit of decomposition of a
  computational task.  DEs are virtual units, not necessarily having 
  a 1-to-1 correspondence to the Persistent Execution Threads (PETs)
  of a VM or the physical Processing Elements (PEs) in the underlying
  physical machine. Consequently there are no restrictions on the 
  number of DEs that can be created. The application writer may chose
  the number of DEs to best match the computational problem and the
  employed algorithm.  A DELayout assigns a topology to
  Decomposition Elements.
  See also \htmlref{DELayout}{glos:DELayout}.

\label{glos:DeepObjects}
\item[Deep object] In an environment
  in which the calling and implementation language of a library are
  different, deep objects are defined as those whose memory is 
  allocated by the implementation language. 
  See also \htmlref{shallow object}{glos:ShallowObjects}. 

\label{glos:DistGrid}
\item[Distributed Grid]
  DistGrid is the ESMF class that defines the 
  decomposition of a Grid's global index space across a DELayout.
  DistGrid objects are contained in an ESMF Grid. 
  See also \htmlref{Grid}{glos:Grid}, \htmlref{DELayout}{glos:DELayout}.

\label{glos:Distribution}
\item[Distribution] The function that expresses
  the relationship between the indices in a Distributed Grid and the elements 
  in a DELayout. See also \htmlref{Distributed Grid}{glos:DistGrid}, 
  \htmlref{DELayout}{glos:DELayout}. 

\label{glos:DomainDecomp}
\item[Domain decomposition] The act of grid 
  distribution: creating a DistGrid, and associating grid points with 
  the DistGrid.  The dimensionality of the domain decomposition is the 
  same as the dimensionality of the associated DistGrid.

\label{glos:Exact}
\item [Exact] The word exact is used
  to denote entities, such as time instants and time intervals, for 
  which truncation-free arithmetic is required. 

\label{glos:ExchangeGrid}
\item[Exchange grid] A grid whose vertices are
  formed by the intersection of the vertices of two overlying grids.  Each 
  cell in the exchange grid overlies exactly one cell in each grid of the 
  exchange. See also \htmlref{grid}{glos:Grid}, \htmlref{cell}{glos:Cell}.

\label{glos:EP}
\item[Exchange Packets] Exchange Packets are a private
  ESMF class that contains data in an optimal form for data transfers.

\label{glos:ExcDomain}
\item[Exclusive domain] For a given DE, the 
  set of data points that are not replicated on any other DE.  See also 
  \htmlref{total domain}{glos:TotDomain},
  \htmlref{computational domain}{glos:CompDomain}, \htmlref{halo}{glos:Halo}.

\label{glos:Exec} 
\item[Executable] 
  A program that is under independent control by the operating 
  system.

\label{glos:ExportState} 
\item[Export State] 
  The data and metadata that 
  a component can make available for exchange with other components. 
  This may be data at a physical boundary (e.g land-atmosphere interface) 
  or in other cases, it might be the entire model state.  
  See also \htmlref{State}{glos:State}, \htmlref{import State}{glos:ImportState}.

\label{glos:Field} 
\item[Field] The ESMF Field class represents a tangible or derived quantity
defined within a continuous region of space.  The Field class includes
the physical grid associated with the quantity and a decomposition
that specifies how data associated with points in the 
physical grid are distributed in computer memory and/or how computational 
work is divided among threads.  A Field also includes a specification 
of gridpoint staggering and any metadata necessary for a full description
of its data.  See also \htmlref{Grid}{glos:Grid}.

\label{glos:Framework} 
\item[Framework] We use the term framework to 
  refer to a structured collection of software building blocks that can be used 
  and customized to develop components, assemble them into an application, and 
  run the application.

\label{glos:GenericComp} 
\item[Generic component] A generic component
  is one supplied by the framework.  The user is not expected to 
  customize or otherwise modify it.  ESMF does not currently contain any
  generic components.  See also \htmlref{user component}{glos:UserComp}, 
  \htmlref{component}{glos:Component}. 

\label{glos:GenericTrans} 
\item[Generic transform] A generic transform 
  is an operation supplied by the framework, for example, a method 
  that converts gridded data from one supported grid and/or 
  decomposition to another using a specified technique.  See also \htmlref{user 
  transform}{glos:UserTrans}.

\label{glos:GlobDomain}
\item[Global domain] 
  A global domain refers to the full extent of a DELayout or Grid.

\label{glos:GlobReduction} 
\item[Global reduction] 
  Reduction operations (sum, max, min, etc.) that condense data distributed
  over a \htmlref{global domain}{glos:GlobDomain}.
  See also \htmlref{global broadcast}{glos:GlobBroadcast}.

\label{glos:GlobBroadcast}
\item[Global broadcast] 
  Scatter operations on data distributed over a 
  \htmlref{global domain}{glos:GlobDomain}.
  See also \htmlref{global reduction}{glos:GlobReduction}.

\label{glos:Gregorian calendar}
\item[Gregorian]
  The Gregorian calendar is the most widely used calendar
in the world.  The calendar's zeroth year is at the
birth of Jesus Christ.  Years after the origin (anno Domini,
or AD) are positive, and before (Before Christ, or BC) are
negative.  Several corrections (leap year, 100 year, 400 year) 
are necessary to keep the calendar aligned with solar cycles.
  See also \htmlref{Calendar}{glos:Calendar}.

\label{glos:GRIB}
\item[GRIB]
  The GRid in Binary Data format from the World Meteorological Organization.
This format is frequently used by operational weather centers.  See the
\htmladdnormallink{GRIB}{http://www.wmo.ch/pages/prog/www/WDM/Guides/Guide-binary-2.html} and
\htmladdnormallink{GRIB2}{http://www.wmo.ch/web/www/DPS/grib-2.html} home pages.

\label{glos:Grid} 
\item[Grid] 
  The discrete division of space associated with
  a particular coordinate system.  The ESMF Grid class contains 
  coordinate, domain decomposition, and memory 
  organization information required to manipulate 
  Fields, as well as to create and execute Grid transforms. 
  See also \htmlref{Distributed Grid}
  {glos:DistGrid}, \htmlref{DELayout}{glos:DELayout}.

\label{glos:GridStagger} 
\item[Grid staggering] 
  A descriptor of relative locations
  of scalar and vector data on a structured grid. On different
  staggered grids, vector data may lie at cell faces or vertices,
  while scalar data may lie in the interior. 

\label{glos:GridTopo} 
\item[Grid topology] 
  Description of data connectivities for a grid.

\label{glos:GridUnion} 
\item[Grid union] 
  The formation of a new grid
  by taking the union of the vertices of two input grids.
  See also \htmlref{Grid}{glos:Grid}. 

\label{glos:GridComp}
\item[Gridded Component] 
  An ESMF class that represents a component that is associated with one 
  or more grids.  No requirements 
  may be placed on the physical content of a Gridded Component's data or 
  on the nature of its computations. See also \htmlref{component}{glos:Component},
  \htmlref{Coupler Component}{glos:Coupler}. 

\label{glos:Halo} 
\item[Halo] 
  For a given DE, a halo is a set of data points from the computational 
  domains of neighboring DEs that are replicated locally for computational
  convenience.  A halo can be defined as all the data points 
  in a DE's total domain excluding those in its computational domain. 
  See also \htmlref{computational domain}{glos:CompDomain}, \htmlref{total domain}
  {glos:TotDomain}, \htmlref{exclusive domain}{glos:ExcDomain}.

\label{glos:HaloUpdate}
\item[Halo update] 
  A halo update operation involves synchronization of the values of some 
  or all halo points with the current values of those points on other DEs.
  See also \htmlref{halo}{glos:Halo}.

\label{glos:ImportState} 
\item[Import State] 
  The data and metadata 
  that a component requires from other components in order to run.  
  See also \htmlref{State}{glos:State}, \htmlref{export State}{glos:ExportState}.

\label{glos:Index} 
\item[Index] 
  An integer value associated with a set of coordinates.

\label{glos:IndexSpace} 
\item[Index space] 
  The space implied 
  by a set of indices.  An index space has a defined dimensionality and 
  connectivity.

\label{glos:IndexSpaceLoc} 
\item[Index space location] 
  A location within an index space.  An index space location may be fractional.
  See also \htmlref{physical location}{glos:PhysLoc}.

\label{glos:Instantiate}
\item[Instantiate] 
  To create an actual instance of a software class.  For example, each 
  variable of derived type Field in an ESMF Fortran application is an 
  instance of the Field class.

\label{glos:Interface}
\item[Interface] 
  Used generally to refer to a set of operations that characterize 
  the behavior of a class or a component.  Also used to refer to the
  name and argument list of a particular method.

\label{glos:JMC} 
\item[Joint Milestone Codeset(JMC)] 
  Joint Milestone Codeset.  This is the set of climate, weather and
  data assimilation applications used as ESMF testbeds 
  during the initial NASA-funded phase of ESMF development.

\label{glos:JST}
\item[Joint Specification Team(JST)]
  The JST is the body of developers and users who collaborate
  to create the ESMF software.  The main form of communication for 
  the JST is the weekly telecon.  Terms of Reference are in the \htmladdnormallink{ESMF Project Plan}{http://www.earthsystemmodeling.org/management/}.

\label{glos:LocalArray}
\item[LocalArray]
  A LocalArray is the portion of an ESMF Array that resides on a 
  particular DE.  See also \htmlref{Array}{glos:Array}.

\label{glos:LocalTile}
\item[LocalTile]
  A LocalTile is the portion of a grid Tile that resides on a 
  particular DE.  See also \htmlref{Tile}{glos:GridTile}.

\label{glos:LocStream} 
\item[Location Stream] 
  An ESMF class that represents
  a list of locations with no assumed relationship between these locations.  
  The elements of a Location Stream are not assumed to share the same 
  metadata. Location Streams are not yet implemented.
  See also \htmlref{background grid}{glos:BackGrid}.

\label{glos:LogRectGrid} 
\item[Logically rectangular grid] 
  A grid in which a set of coordinates (x,y,z, ...) in physical
  space can be mapped one-to-one to a set of regularly spaced 
  points (i,j,k, ...) in a rectangular logical space, preserving
  proximate relationships.  See also \htmlref{Grid}{glos:Grid}.

\label{glos:LooseFieldBundle} 
\item[Loose FieldBundle] 
  A loose FieldBundle 
  is an ESMF FieldBundle object that contains fields whose data is 
  not contiguous in memory.  See also \htmlref{FieldBundle}{glos:FieldBundle},
  \htmlref{packed FieldBundle}{glos:PackedFieldBundle}.

\item[Machine model] A generic representation of the computing 
  platform architecture.

\label{glos:Mask} 
\item[Mask] 
  A data field marking a span within a larger data field.

\label{glos:MemDomain} 
\item[Memory domain] 
  The portion of memory associated with the data on a given DE.  
  The memory domain is always at least 
  as large as the total domain.  See also \htmlref{total domain}{glos:TotDomain}.

\label{glos:MosaicGrid}
\item[Mosaic grid]
  A mosaic grid is composed of multiple logically rectangular
  grid tiles that are connected at their edges, for example, a cubed sphere
  grid.  See also \htmlref{grid tile}{glos:GridTile}.

\label{glos:MPMD} 
\item[MPMD] 
  Multiple Program Multiple Datastream.
  Multiple executables, any of which could itself be an SPMD
  executable, executing independently within an application. 
  See also \htmlref{SPMD}{glos:SPMD}.

\label{glos:Namelist}
\item[Namelist]
An I/O feature supported by Fortran that
defines a structured syntax for creating text files of initial variable
settings and defines language features for compactly reading the files.
The syntax for Namelist files can be found in most Fortran manuals 
and tutorial texts.

\label{glos:NetCDF}
\item[NetCDF]
  Network Common Data Form.  This is a popular I/O library and data format
in the Earth sciences. See \htmladdnormallink{NetCDF home page}{http://www.unidata.ucar.edu/software/netcdf/}. 

\label{glos:Node} 
\item[Node] 
  A node is a set of computational resources
  that is typically located in close proximity on a computing platform
  and that is associated with a single shared memory buffer.

\label{glos:NoLeap} 
\item [No-leap calendar] 
  In this calendar every year uses the same months 
  and days per month as in a non-leap year of a Gregorian calendar.  See
  also \htmlref{Calendar}{glos:Calendar}, \htmlref{360-day calendar}{glos:360DayCal}.

\label{glos:PackedFieldBundle} 
\item[Packed FieldBundle] 
  A packed FieldBundle is an 
  ESMF FieldBundle object that contains
  a data buffer with field data arranged contiguously in memory. See 
  also \htmlref{FieldBundle}{glos:FieldBundle}, \htmlref{loose FieldBundle}{glos:LooseFieldBundle}.

\label{glos:Parallel}
\item[Parallel execution]
  The term parallel execution refers to the execution of a software
  application on more than one \htmlref{PE}{glos:PET}.
  See also \htmlref{serial}{glos:Serial}.

\label{glos:PE} 
\item[PE] 
  Short for \htmlref{Processing Element}{glos:Processing_Element}.

\label{glos:PET} 
\item[PET] 
  Short for \htmlref{Persistent Execution Thread}{glos:PermET}.

\label{glos:PermET} 
\item[Persistent Execution Thread (PET)] 
  Provides a
  path for executing an instruction sequence. A PET has a lifetime at least 
  as long as the associated data objects. The PET is a key abstraction 
  used in the ESMF \htmlref{Virtual Machine}{glos:VM}.

\label{glos:PhysLoc} 
\item[Physical location] 
  A point in physical space to which a data point pertains.  See also
  \htmlref{index space location}{glos:IndexSpaceLoc}.   

\label{glos:Platform} 
\item[Platform] 
  The processor hardware, operating system, compiler and
  parallel library that together form a unique compilation target.

\label{glos:Processing_Element}
\item[Processing Element (PE)] 
  A Processing Element (PE) is the smallest physical processing unit available
  on a particular hardware platform.

\label{glos:RectilinearGrid}
\item[Rectilinear grid]
  A rectilinear grid is a logically rectangular grid in which 
  the coordinates in physical space can be fully specified by
  the spacing of grid points along each grid axis.  The
  gridpoints are located where the coordinate values intersect.
  The spacing along each axis may vary.  
  See also \htmlref{logically rectangular grid}{glos:LogRectGrid},
  \htmlref{Uniform grid}{glos:UniformGrid}, \htmlref{Curvilinear grid}
  {glos:CurvilinearGrid}.  

\label{glos:Scheduler} 
\item[Scheduler] 
  An operating system component 
  that assigns system resources (processors, memory, CPU time, 
  I/O channels, etc.) to executables.

\label{glos:Search} 
\item[Search]
  {\tt Search} refers to the process of determining which processors must
  exchange data (and how much) when regridding between decomposed grids.
  See also \htmlref{sweep}{glos:Sweep}.

\label{glos:SequentialExecution}
\item[Sequential execution] 
  Sequential execution of model components describes the case in which 
  one component waits for another to finish before it begins
  to run.  Components executing sequentially may be in the same or 
  different executables and may have coincident or non-overlapping 
  memory distributions.  See \htmlref{Concurrent execution}
  {glos:ConcurrentExecution}.

\label{glos:Serial}
\item{Serial Execution}
  The term serial execution refers to the execution of a software
  application on only one PET.  See also \htmlref{parallel execution}
  {glos:Parallel}.

\label{glos:ShallowObjects} 
\item[Shallow object] 
  In an environment
  in which the calling and implementation language of a library are
  different, shallow objects are defined as those whose memory is 
  allocated by the calling language. 
  See also \htmlref{deep object}{glos:DeepObjects}.

\label{glos:Span} 
\item[Span] 
  The physical extent associated with a grid.

\label{glos:SPMD} 
\item[SPMD] 
  Single Program Multiple Datastream. 
  A single executable, possibly with many components (representing 
  for example the atmosphere, the ocean, land surface) executing 
  serially or concurrently. See also \htmlref{MPMD}{glos:MPMD}. 

\label{glos:State} 
\item [State] 
  The ESMF State class may 
  contain Arrays, FieldBundles, Fields, or other States.  It is used to 
  transfer data between components.  See also \htmlref{import State}
  {glos:ImportState}, \htmlref{export State}{glos:ExportState}.

\label{glos:Sweep} 
\item[Sweep]
  {\tt Sweep} refers to the regridding process of looping through lists of cells
  from one grid, hunting for interactions with a specified point or subsegment
  from the other grid.  The type of interaction depends on the regrid method
  and is either an intersection with an identified subsegment or containment
  of an identified point.  The limitation of the range of cells that must be
  examined is also considered part of the sweep algorithm.
  See also \htmlref{search}{glos:Search}.

\label{glos:SysTime} 
\item [System time] 
  Time spent doing system tasks 
  such as I/O or in system calls.  May also include time spent running 
  other processes on a multiprocessor system. See also \htmlref{user 
  time}{glos:UserTime}, \htmlref{wall clock time}{glos:WallClockTime}.

\label{glos:TaskParallel}  
\item[Task parallel] 
  The quality of an application that allows
  different calculations to be performed by different processors at the same 
  time on what are usually different data sets.  Large-scale task parallelism 
  is often associated with multi-component applications in which each component
  represents a separate domain or function.  Task parallel applications 
  may be run with components executing either sequentially or concurrently, 
  and either in a SPMD or MPMD mode. See also 
  \htmlref{data parallel}{glos:DataParallel}, 
  \htmlref{SPMD}{glos:SPMD}, \htmlref{MPMD}{glos:MPMD}, 
  \htmlref{sequential execution}{glos:SequentialExecution}, 
  \htmlref{concurrent execution}{glos:ConcurrentExecution}.

\label{glos:GridTile}
  Some grids used in Earth system modeling, such as cubed sphere grids, are
  most naturally represented
  as a set of logically rectangular grids that are connected at their 
  edges.  Following V. Balaji [2006] we refer to each of the 
  logically rectangular grids in a composite grid, or mosaic grid, as a
  Tile.  See also \htmlref{mosaic grid}{glos:MosaicGrid}, 
  \htmlref{LocalTile}{glos:LocalTile}.

\label{glos:TimeInstant}
\item [Time] 
  Time is an ESMF class that is made up of a time and date and an 
  associated calendar. It may include a time zone.
  \emph{Jan 3rd 1999, 03:30:24.56s, UTC} is one example of a Time.
  See also \htmlref{Calendar}{glos:Calendar}.

\label{glos:TimeInterval} 
\item [Time Interval] 
  Time Interval is an ESMF class that represents the
  period between any two time instants, measured in units, such as days, 
  seconds, and fractions of a second.  The periods \emph{2 days and 10 seconds}, 
  \emph{86400 and 1/3 seconds} and \emph{31104000.75 seconds} are all 
  possible values for Time Intervals.  
  Mathematical operations such as addition, multiplication, and subdivision 
  can be applied to Time Intervals, and they can have negative values. 
  See also \htmlref{Time}{glos:TimeInstant}

\label{glos:TotDomain} 
\item[Total domain] 
  For a given DE, the entirety 
  of the data points allocated, included replicated points from neighboring
  DEs.  See also \htmlref{computational domain}{glos:CompDomain}, 
  \htmlref{exclusive domain}{glos:ExcDomain}, \htmlref{halo}{glos:Halo}

\label{glos:UniformGrid}
  A logically rectangular grid in which the coordinates in physical 
  space can be completely specified by the two sets of coordinates
  that define the opposing corner points of the physical span.  The coordinates
  of each point in physical space can be obtained by interpolating from
  the corner points, using the evenly spaced logical grid to specify 
  evenly spaced grid point locations.  See also 
  \htmlref{logically rectangular grid}{glos:LogRectGrid},
  \htmlref{Rectilinear grid}{glos:RectilinearGrid}, \htmlref{Curvilinear grid}
  {glos:CurvilinearGrid}.  

\label{glos:UserComp} 
\item[User component] 
  A component that is customized or
  written by the user.  All ESMF components are currently user components.
  See also \htmlref{generic component}{glos:GenericComp}.

\label{glos:UserTime} 
\item[User time] 
  Processor time actually spent executing 
  a PET's code. See also \htmlref{system time}{glos:SysTime}, 
  \htmlref{wall clock time}{glos:WallClockTime}.

\label{glos:UserTrans} 
\item[User transform] 
  A user-supplied 
  method that is used to extend framework capabilities beyond generic 
  transforms. See also \htmlref{generic transform}{glos:GenericTrans}. 

\label{glos:VAS}
\item[Virtual Address Space (VAS)] A term that refers to 
  the address space in which the computer memory is represented and
  becomes accessible to an executing PET.

\label{glos:VM} 
\item[VM] 
  Short for \htmlref{Virtual Machine}{glos:VMachine}.

\label{glos:VMachine} 
\item[Virtual Machine (VM)] 
  An ESMF class that abstracts hardware and 
  operating system details. The VM's responsibilities are resource management
  and topological description of the underlying compute resources in terms of 
  \htmlref{PETs}{glos:PET}. In addition the VM provides a transparent, low level
  communication API. 

\label{glos:WallClockTime} 
\item [Wall clock time] 
  Elapsed real-world time 
  (i.e. difference between start time minus stop time).
  See also \htmlref{system time}{glos:SysTime}, \htmlref{user time}{glos:UserTime}.

\end{description}
