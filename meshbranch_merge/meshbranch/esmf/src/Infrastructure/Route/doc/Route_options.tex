% $Id: Route_options.tex,v 1.6 2008/04/02 20:42:57 cdeluca Exp $

\subsubsection{ESMF\_RouteOptions}

\label{opt:routeopt}
{\sf DESCRIPTION:\\}
Specifies control options when executing the communication
represented by a Route object.  Normally these do not need to
be set by the user, but can be specified if the best communication strategy
is known in advance.   The synchronous and asychronous options 
are mutually exclusive; and the other packing options are also
mutually exclusive.  Setting the Route options is "sticky"; it
maintains the last value set until explicitly changed.

Note that these options control the internal execution of a single
set of communications represented by a Route object and do not affect
the user level behavior at all.  For example,
the asynchronous option does not cause the user level entry point to
return sooner; it means the route will queue all communication requests
first and then go back and check for completion in an internal loop.

Valid values are:
\begin{description}
    \item [ESMF\_ROUTE\_OPTION\_ASYNC]
	Use an internal asynchronous strategy to execute the Route.
    \item [ESMF\_ROUTE\_OPTION\_SYNC]
	Use an internal synchronous strategy to execute the Route.
    \item [ESMF\_ROUTE\_OPTION\_PACK\_PET]
        Pack all data from or to another PET into a single buffer
        when sending or receiving.
    \item [ESMF\_ROUTE\_OPTION\_PACK\_XP]
        Pack all data from each non-contiguous exchange packet 
        into a single buffer when sending or receiving.
    \item [ESMF\_ROUTE\_OPTION\_PACK\_NOPACK]
        Do no buffering; send each contiguous run of data as a distinct
        communications operation.
    \item [ESMF\_ROUTE\_OPTION\_PACK\_VECTOR]
        Use the MPI type vector interfaces to send non-contiguous data
        which has regular strides when sending or receiving.
    \item [ESMF\_ROUTE\_OPTION\_PACK\_BUFFER]
        When multiple data addresses are sent to the Route routines (for
        example, identical {\tt ESMF\_Fields} from an {\tt ESMF\_FieldBundle}),
        this flag controls whether to pack the buffers together or send
        them separately.  
    \item [ESMF\_ROUTE\_OPTION\_DEFAULT]
	Use the system default for communication, which is the combination of 
        {\tt ESMF\_ROUTE\_OPTION\_PACK\_BUFFER},
        {\tt ESMF\_ROUTE\_OPTION\_PACK\_PET}, and
        {\tt ESMF\_ROUTE\_OPTION\_SYNC}.
\end{description}







