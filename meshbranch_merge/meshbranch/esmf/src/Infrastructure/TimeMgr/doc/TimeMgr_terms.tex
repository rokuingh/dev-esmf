% $Id: TimeMgr_terms.tex,v 1.1 2002/10/08 17:16:04 eschwab Exp $
\label{sec:terms}

\begin{description}

\item [time instant] \label{glos:timeinstant}
Generic name for an absolute time and date specification. A time instant is made 
up of a time and date and an associated calendar. It may include a time zone.
``Jan 3rd 1999, 03:30:24.56s, UTC'' is one example of a time instant.

\item [time interval] \label{glos:timeinterval} A time interval is the
period between any two time instants, measured in units, such as days, 
seconds, and fractions of a second, that are not associated with a specific
calendar.  Time intervals may be negative.  The periods 2 days and 10 seconds, 
86400 and 1/3 seconds and 31104000.75 seconds are all examples of time intervals.  
Mathematical operations such as addition, multiplication and subdivision 
can be applied to time intervals.
              
\item [alarm] \label{glos:alarm} An alarm is an event 
that occurs at a particular time (or set of times).  It is like an
alarm on a real alarm clock except that in order to determine whether 
it is "ringing", an alarm is "read" by an explicit application action.

\item [clock] \label{glos:clock} A clock tracks the passage of time and 
reports the current time instant, like a real clock.  However, most clocks 
used in ESMF components have a key difference to a real clock. Clocks 
in an ESMF component are generally stepped forward by the component, as an 
explicitly coded step within the overall component.

\item [calendar interval] \label{glos:timeperiod} A period of time specified
in calendar-based units that may be used to increment or decrement time instants.  
One year and three months is an example of a calendar interval.  Since 
mathematical operations involving calendar intervals may be ambiguously 
defined -- for example, incrementing January 31 in the Gregorian calendar by 
one month -- default behavior must be carefully specified.  

\item [day of year] \label{glos:dayofyear} The day number in the calendar year. 
January 1 is day 1 of the year. Day of year expressed in a floating point 
format is used to express the day number plus the time of day. 
For example, assuming a Gregorian calendar:

\begin{tabular}{ll}
{\bf date}              & {\bf day of year} \\
\hline 
10 January 2000, 6Z     & 10.25 \\
31 December 2000, 18Z   & 366.75 
\end{tabular}

\item [exact] \label{glos:exact} The word {\em exact} is used in this document
to denote entities (time instants and time intervals) for which truncation-free 
arithmetic is required. An {\em exact} time operation (increment, decrement, 
difference), when applied exclusively to {\em exact} objects, produces a
truncation-free result.

\item [no-leap calendar] \label{glos:noleap} Every year uses the same months 
and days per month as in a non-leap year of a Gregorian calendar.

\end{description}











