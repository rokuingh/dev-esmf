% $Id: TimeMgr_sumreq.tex,v 1.2 2003/07/24 15:50:50 cdeluca Exp $

The basic capabilities required by the time manager are satisfied by
the concepts of time intervals, time instants, clocks and alarms, as defined 
in the glossary.  

Time intervals and time instants are the computational building blocks of the \funcname 
library.  Time intervals, which are time periods independent of any calendar, support 
operations such as add, subtract, compare size, reset value, copy value, and 
subdivide by a scalar.  Time instants, which are moments in time associated 
with specific calendars, can be incremented or decremented by time intervals, compared 
to see which of two time instants is later, differenced to obtain the interval
between two time instants, copied, reset, and manipulated in other useful ways.
Time instants support a host of different queries, both for values of individual time instant 
components such as year, month, day and second, and for derived values such 
as day of year, middle of current month and Julian day.  It is also possible 
to retrieve the value of the hardware realtime clock in the form of a time instant.

Since climate modeling, numerical weather prediction and other 
Earth system applications have widely varying time scales and require different sorts
of calendars, the \funcname must provide a wide range of time specifiers, spanning 
nanoseconds to years.  The set of supported calendars includes Gregorian, no-leap,
Julian, and 360-day.  The \funcname also supports a user-specified calendar.

Although it is possible to repeatedly step a time instant forward by a time interval using 
arithmetic on these basic types, it is useful to identify a higher-level concept 
that encapsulates this function.  We refer to this capability as a clock, and include 
in its required features the ability to store reference times such as the start and
stop time instants of a model run, to check when time advancement should cease, and to query 
the value of quantities such as the previous and current time instants.  The \funcname 
must include methods that return a flag value when a periodic or unique event has taken 
place; we refer to these as alarms.  Applications may require temporary 
or multiple clocks and alarms.  

For operations on time types, finite precision arithmetic that has defined semantics 
for both floating point and integer operands is required. Arithmetic based on rational 
fractions, with support for arbitrarily accurate drift-free 
(i.e. {\it exact}) clocks, is desired to extend the capabilities of target
applications.

The time manager must satisfy the framework-wide requirements for the ESMF described 
in the {\it ESMF General Requirements} (see~\cite{ESMFGenReq}), including requirements for 
supported platforms, robust error handling and real and integer precision.

















