% $Id: TimeInterval_desc.tex,v 1.12 2009/02/13 15:48:32 murphysj Exp $
\label{sec:TimeInterval}
A TimeInterval represents a period between time instants.  
It can be either positive or negative.  Like the Time interface, 
the TimeInterval interface is designed so that you can choose 
one or more options from a list of time units in order 
to specify a TimeInterval.
See Section ~\ref{subsec:Time Instants and TimeIntervals}, 
Table ~\ref{table:timeOpts} for the available options.

There are TimeInterval methods defined for setting and getting 
a TimeInterval, for incrementing and decrementing a TimeInterval 
by another TimeInterval, and for multiplying and dividing 
TimeIntervals by integers, reals, fractions and other TimeIntervals.  
Methods are also defined to take the absolute value and negative 
absolute value of a TimeInterval, and for comparing the length of two
TimeIntervals.

The class used to represent time instants in ESMF is Time,
and this class is frequently used in operations along with 
TimeIntervals.  For example, the difference between two
Times is a TimeInterval.  

When a TimeInterval is used in calculations that involve an absolute 
reference time, such as incrementing a Time with a TimeInterval, calendar 
dependencies may be introduced.  The length of the time period that the 
TimeInterval represents will depend on the reference Time and the 
standard calendar that is associated with it.  The calendar dependency becomes 
apparent when, for example, adding a TimeInterval of 1 day to the Time 
of February 28, 1996, at 4:00pm EST.  In a 360 day calendar, the 
resulting date would be February 29, 1996, at 4:00pm EST.  In a no-leap 
calendar, the result would be March 1, 1996, at 4:00pm EST.

TimeIntervals are used by other parts of the ESMF timekeeping
system, such as Clocks (Section~\ref{sec:Clock}) and Alarms 
(Section~\ref{sec:Alarm}).





